\chapter{Introduzione}\label{cap:introduzione}
\section{Contesto}\label{sez:contesto}
Al giorno d'oggi la presenza online per gli artisti \`e fondamentale per farsi notare in un mercato concorrenziale come quello dell'arte. La presenza online \`e un fattore chiave per un artista che vuole intraprendere una carriera professionale in quanto facilita la costruzione di relazioni con altri artisti e appassionati, permettendo di raggiungere un pubblico pi\`u ampio rispetto ai metodi  tradizionali, come la partecipazione a mostre d'arte o riviste. Il vantaggio principale della presenza online \`e sicuramente la capacit\`a di attirare potenziali clienti e ricevere opportunit\`a di lavoro.

Oltre alla presenza sui social media, \`e possibile costruire una solida presenza online attraverso la creazione di un portfolio artistico digitale. Nell'accezione pi\`u comune del termine, per portfolio si intende una raccolta ben organizzata dei lavori svolti da una persona durante il corso della propria carriera in modo tale da dar prova delle proprie capacit\`a e competenze.
Avere un portfolio in formato digitale, reperibile su un sito web o su una piattaforma online, permette di raggiungere un pubblico globale, rispetto al formato cartaceo, che \`e limitato nella sua diffusione e visibilit\`a. Un portfolio, quindi, permette ad un artista di esibire in maniera strutturata, attraverso l'utilizzo di gallerie e o categorie ben distinte tra loro, le proprie opere, creando una vetrina virtuale che evidenzia il proprio percorso artistico e il proprio stile.

La sezione di presentazione \`e una parte immancabile in un portfolio. Per avere una buona presenza online esibire i propri lavori non basta, \`e necessario includere anche una breve biografia nella quale si va a descrivere la propria personalit\`a e formazione. Grazie a ci\`o \`e possibile creare una relazione con chi visiona il portfolio e rafforzare la propria credibilit\`a. Nella sezione di presentazione saranno inoltre presenti le informazioni di contatto, in modo tale che potenziali clienti possano mettersi in contatto con l'autore del portfolio.

In conclusione, la presenza online e la creazione di un portfolio sono strumenti fondamentali per promuovere la propria carriera artistica ed entrare in contatto con professionisti del settore dell'arte e ottenere opportunit\`a di lavoro.

\section{Obiettivi della piattaforma}\label{sez:obiettivi_della_piattaforma}
Data l'enorme importanza della presenza online per gli artisti, si \`e deciso di sviluppare una piattaforma che consenta loro di realizzare e gestire un proprio portfolio artistico digitale in modo semplice e intuitivo.

La piattaforma offre agli artisti la possibilit\`a di condividere i propri lavori su Internet, organizzandoli in gallerie. Le gallerie consentono di distinguere i lavori in categorie specifiche, rendendo pi\`u facile per gli altri utenti navigare e trovare i lavori che li interessano. A tal fine, ogni galleria sar\`a accompagnata, oltre che da un titolo, da una descrizione in modo tale da esplicitare il contesto e i criteri delle opere presenti in essa, come il tema, la tecnica utilizzata o il periodo temporale nel quale \`e stata realizzata. Similmente, anche le opere all'interno della galleria saranno dotate di proprio titolo e descrizione in modo tale da fornire ulteriori dettagli ed esplicitare il loro significato.

Oltre alle sezioni che rappresentano le gallerie, in ogni portfolio sar\`a presente la sezione di presentazione che includer\`a in breve la sua biografia e il suo indirizzo e-mail, in modo tale che possa essere contattato da chi visiona il suo portfolio.

L'obiettivo principale della piattaforma \`e, quindi, di consentire agli artisti di creare una solida presenza online, fornendo gli strumenti per creare un portfolio digitale in modo tale da promuovere la loro carriera.

\section{Struttura dei prossimi capitoli}\label{sez:struttura_ dei_prossimi_capitoli}
I capitoli della tesi successivi a questo sono strutturati nel seguente modo:
\begin{itemize}
	\item Il secondo capitolo, \textit{``\nameref{cap:specifica_dei_requisiti}''}, presenta il documento di ``specifica dei requisiti'' nel quale vengono descritte le funzionalit\`a e le caratteristiche che la piattaforma dovr\`a possedere per soddisfare gli obiettivi individuati durante la fase di analisi.
	\item Il terzo capitolo, \textit{``\nameref{cap:design}''}, tratta della progettazione della piattaforma, spiegando le scelte di progetto riguardo l'architettura del sistema e gli strumenti utilizzati per realizzarla, e della sua relativa base di dati.
	\item Il quarto capitolo, \textit{``\nameref{cap:implementazione}''}, \`e dedicato alla fase di sviluppo della piattaforma, illustrando come sono stati configurati e sviluppati i progetti back-end e front-end.
	\item Il quinto ed ultimo capitolo, \textit{``\nameref{cap:conclusioni}''}, contiene i risultati raggiunti e indica gli eventuali sviluppi futuri.
\end{itemize}











