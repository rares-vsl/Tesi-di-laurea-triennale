\chapter{Specifica dei requisiti}\label{cap:specifica_dei_requisiti}
La prima fase del processo di realizzazione del progetto \`e certamente l'analisi dei requisiti, nella quale vengono identificate tutte quelle che sono le funzionalit\`a e le caratteristiche della piattaforma. Ci\`o deve avvenire in maniera esaustiva, al fine di progettare la piattaforma in modo tale che possa soddisfare ogni esigenza degli utenti. Da questa fase si ricava un documento, denominato ``specifica dei requisiti'' e riportato in questo capitolo, nel quale vengono descritti, utilizzando il linguaggio naturale, tutti i requisiti individuati durante l'analisi.

\section{Autenticazione}\label{cap:autenticazione}
I requisiti di autenticazione riguardano principalmente la fasi di registrazione, nella quale un utente crea un nuovo account, e la gestione dell'accesso alla piattaforma tramite le credenziali scelte durante la registrazione.

\subsection{Registrazione}
Ogni nuovo utente che vuole registrarsi alla piattaforma dovr\`a compilare un modulo per fornire le informazione personali (nome, username) e le credenziali di accesso alla piattaforma (indirizzo e-mail, password). L'username permette di identificare univocamente l'utente all'interno piattaforma.

Una volta inviato tale modulo ricever\`a, nella casella postale dell'indirizzo e-mail fornito, un link per verificare il proprio indirizzo e-mail.

\subsection{Accesso}
Per accedere alla piattaforma, ogni utente registrato deve compilare un modulo per fornire le sue credenziali di accesso (e-mail, password). Inoltre pu\`o indicare se vuole rimanere connesso alla piattaforma anche dopo la scadenza della sessione, senza dover reinserire le credenziali di accesso.

Nel caso in cui non si dovesse ricordare la password potr\`a fornire l'indirizzo e-mail, con il quale si \`e registrato alla piattaforma, cos\`i da ricevere per e-mail un link per reimpostarla.
\section{Impostazioni}
La piattaforma deve prevedere una sezione dedicata alle impostazioni nella quale ogni utente potr\`a modificare le proprie informazioni.
\subsection{Account}
Nella sezione dedicata all'account, sar\`a presente un modulo per permettere all'utente di modificare il proprio username e il proprio indirizzo e-mail.

\subsection{Profilo}
Nella sezione dedicata al profilo, sar\`a presente un modulo per permettere all'utente di modificare il proprio nome e aggiornare o rimuovere la propria immagine di profilo.

\subsection{Password}
Nella sezione dedicata alla password, sar\`a presente un modulo per permettere all'utente di impostare una nuova password, per fare ci\`o, dovr\`a anche fornire la sua password attuale.

\subsection{Chiusura account}
Tra le voci del men\`u delle impostazioni sar\`a presente l'opzione per chiudere il proprio account. Per compiere tale azione l'utente dovr\`a fornire nuovamente la sua password attuale.

Una volta chiuso l'account, l'utente verr\`a disconnesso dalla piattaforma e avr\`a 60 giorni per effettuare nuovamente l'accesso e ripristinare il proprio account.
\section{Gestione del Portfolio}
La piattaforma deve prevedere una sezione dedicata alla gestione dei contenuti presenti nel portfolio dell'utente. Tale sezione prende il nome di CMS (Content Management System) e permetter\`a
di gestire i contenuti delle sezioni del portfolio.
\subsection{Creazione portfolio}
Accedendo alla sezione CMS, se l'utente non ha ancora creato il proprio portfolio ed ha verificato il proprio indirizzo e-mail, verr\`a portato ad una pagina contenente un modulo nel quale dovr\`a fornire il nome del portfolio che vuole creare, il quale identificher\`a univocamente il portfolio all'interno piattaforma.

\subsection{Pagina principale}
La pagina principale del CMS conterr\`a la lista delle sezioni visibili del portfolio e permetter\`a di creare una nuova galleria. Inoltre, per ogni sezione nella lista, sar\`a presente un pulsante per modificare le informazioni di tale sezione.

\subsection{Gestione delle gallerie}
La sezione del CMS dedicata alle gallerie permetter\`a all'utente di creare nuove gallerie e modificare, eliminare ed archiviare le gallerie esistenti. Le gallerie eliminate dall'utente vengono inserite in un cestino in modo tale da essere ripristinate in un secondo momento.

\subsubsection{Gallerie principali}
Nella sezione principale dedicata alle gallerie sar\`a presente la lista delle gallerie visibili e archiviate e, per ogni galleria nella lista, saranno presenti i pulsanti per modificare lo stato di archiviazione, modificare le informazioni ed eliminare tale galleria.

\subsubsection{Creazione}
Nella sezione dedicata alla creazione di una nuova galleria, sar\`a presente un modulo per permettere all'utente di fornire il nome della nuova galleria, il quale dovr\`a essere univoco nel suo portfolio, e una descrizione.

\subsubsection{Modifica}
Il pulsante dedicato alla modifica della galleria porta l'utente nella sezione dedicata alla modifica delle informazioni di tale galleria, nella quale, sar\`a presente un modulo per permettere la modifica del nome e della descrizione. Inoltre, in tale sezione, l'utente potr\`a modificare la posizione della galleria nella barra di navigazione del proprio portfolio.

\subsubsection{Archiviazione}
Il pulsante dedicato allo stato di archiviazione della galleria permette di archiviare, in modo tale che non sia visibile nel proprio portfolio, e disarchiviare tale galleria.

\subsubsection{Eliminazione}
Il pulsante dedicato all'eliminazione della galleria permette di eliminare tale galleria, rimuovendola dalla lista e spostandola nel cestino.

\subsubsection{Cestino}
La sezione dedicata alle gallerie presenta una sotto sezione per la gestione del cestino, nella quale sar\`a presente la lista delle gallerie eliminate e, per ogni galleria nella lista, saranno presenti i pulsanti per ripristinare ed eliminare definitivamente tale galleria.

\subsubsection{Ripristino}
Il pulsante dedicato al ripristino della galleria permette di ripristinare una galleria, rimuovendola dal cestino.

\subsubsection{Eliminazione definitiva}
Il pulsante dedicato all'eliminazione definitiva della galleria permette di eliminare definitivamente le informazioni e il contenuto di una galleria.

\subsection{Gestione dei post}
In ogni galleria, oltre che a visualizzare le relative informazioni della galleria, sar\`a possibile creare nuovi post e modificare ed eliminare i post esistenti. Ogni post rappresenta un'opera dell'artista ed \`e costituito da un media, un titolo ed una descrizione.

All'intero della galleria, quindi, sar\`a presente una lista dei post creati dall'utente e, per ogni post nella lista, saranno presenti i pulsanti per modificare le informazioni ed eliminare tale post.

\subsubsection{Creazione}
Nella sezione dedicata alla creazione di un nuovo post, sar\`a presente un modulo per permettere all'utente di fornire il media, il titolo e la descrizione del post da creare.

\subsubsection{Modifica}
Il pulsante dedicato alla modifica del post porta l'utente nella sezione dedicata alla modifica delle informazioni di tale post, nella quale, sar\`a presente un modulo per permettere la modifica del media, del titolo e della descrizione.

\subsubsection{Eliminazione}
Il pulsante dedicato all'eliminazione del post permette di eliminare definitivamente le informazioni e il contenuto di tale post.


\subsection{About me}
La sezione del CMS dedicata alla presentazione dell'artista prende il nome di ``About me'' e permetter\`a all'utente di presentarsi e fornire i propri contatti. Sar\`a, quindi, presente un modulo per permettere all'utente di fornire una descrizione della propria persona e l'indirizzo e-mail per ricevere contatti.

\subsection{Impostazioni}
La sezione di CMS deve prevedere una sezione dedicata alle impostazioni del portfolio nella quale ogni utente potr\`a modificare le informazioni e lo stato del proprio portfolio.

\subsubsection{Informazioni portfolio}
Nella sezione dedicata alle informazioni del portfolio, sar\`a presente un modulo per permettere all'utente di modificare il nome e l'icona del portfolio.

\subsubsection{Archiviazione}
Tra le voci del men\`u delle impostazioni sar\`a presente l'opzione per archiviare il proprio portfolio, in modo tale che non sia visibile pubblicamente, e disarchiviarlo.

\subsubsection{Eliminazione}
Tra le voci del men\`u delle impostazioni sar\`a presente l'opzione per eliminare il proprio portfolio. Una volta eliminato, l'utente verr\`a portato alla pagina principale della piattaforma.

Il portfolio pu\`o essere ripristinato in qualsiasi momento.


\section{Visualizzazione portfolio}
Ogni utente, autenticato e non, potr\`a visualizzare i portfolio pubblici degli utenti della piattaforma.

Per ogni portfolio verr\`a visualizzato il nome, l'icona e una barra di navigazione per navigare tra le varie gallerie presenti nel portfolio e visualizzare la sezione ``About me'' dedicata alla presentazione dell'autore del portfolio.







