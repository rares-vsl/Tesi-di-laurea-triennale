\chapter{Conclusioni}\label{cap:conclusioni}
L'obiettivo di questa tesi era la progettazione e lo sviluppo di una piattaforma per la gestione di portfolio artistici. Il portfolio \`e uno strumento fondamentale per gli artisti, in quanto offre loro la possibilit\`a di promuovere la propria carriera e costruire una solida presenza online. Il processo di progettazione \`e iniziato partendo dalla stesura del documento di specifica dei requisiti, nel quale sono state descritte le funzionalit\`a e le caratteristiche della piattaforma, per garantire che potesse soddisfare le esigenze degli artisti. In seguito all'analisi dei requisiti, si \`e passati alla progettazione dell'intero sistema che supporter\`a la piattaforma, analizzando attentamente la sua architettura e determinando gli strumenti per la sua realizzazione. La fase di progettazione si \`e conclusa con la progettazione della base di dati che ha permesso di individuare la struttura logica per i dati, in modo che siano facilmente accessibili e gestibili dal sistema. Terminata la fase di progettazione \`e iniziata quella di sviluppo della piattaforma, sviluppando, in modo parallelo, l'applicazione back-end, realizzata con Laravel, e l'applicazione front-end, realizzata con Vue.js e Tailwind CSS.
\section{Risultati ottenuti}
La piattaforma sviluppata rispetta gli obiettivi prefissati e permette la gestione di portfolio artistici in modo semplice e veloce. Un nuovo artista pu\`o iscriversi alla piattaforma tramite la pagina di ``Sign up'' e, dopo aver verificato il suo indirizzo e-mail, potr\`a creare il proprio portfolio.  Nella sezione di CMS gestir\`a le varie sezioni del suo portfolio, quali gallerie e la sezione ``About me'', dedicata alla propria presentazione. In ogni galleria potr\`a aggiungere e gestire i propri lavori rappresentati sotto forma di post dotati di un media, un titolo ed una descrizione.

La piattaforma gestisce appieno le operazioni di creazione, lettura, aggiornamento e rimozione delle risorse, gestendo, inoltre, l'operazione di rimozione parziale di utenti, portfolio e gallerie in modo tale che possano essere ripristinate in un secondo momento.
\section{Sviluppi futuri}
La presenza online per gli artisti, attraverso l'utilizzo di un portfolio digitale, permette di avviare la propria carriera professionale.  Il passo successivo \`e rappresentato dalla possibilit\`a di generare entrate vendendo i propri lavori e ricevendo commissioni.

Gli sviluppi futuri della piattaforma potrebbero, dunque, includere funzionalit\`a per facilitare la gestione di una carriera artistica professionale, come:
\begin{itemize}
	\item Integrazione di funzionalit\`a di e-commerce per consentire agli artisti di vendere le loro opere, e altri prodotti, direttamente dal loro portfolio.
	\item Integrazione di un sistema per la gestione di commissioni per consentire agli artisti di offrire servizi su richiesta.
	\item Integrazione di archiviazione di foto nel cloud per gestire ed accedere alle proprie opere con facilit\`a.
	\item Integrazione di una sezione blog per condividere articoli, notizie, consigli e tutorial relativi alla carriera artistica.
	\item Integrazione di strumenti per la gestione dell'aspetto visivo del portfolio per consentire agli artisti di personalizzare il layout e la disposizione delle loro opere.
\end{itemize}